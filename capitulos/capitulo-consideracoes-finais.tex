\chapter{Considerações Finais}

 Este trabalho propõe um sistema capaz detectar código de barras e números através de uma imagem. Os sistemas de reconhecimento de objetos e aplicação web foram implementados. Como mostrado, a implementação é capaz identificar os códigos de barras, identificar os números e mostrá-los em uma aplicação através de telas e imagens.
 
Ao final do desenvolvimento do projeto e dos resultados apresentados, pode-se concluir que o sistema é eficaz na identificação dos objetos e na gestão do status da corrida, podendo ser um grande aliado na redução da misturas de aço, automatização do processo e na segurança do trabalhador.

Como trabalhos futuros, sugere-se a implementação física do sistema em uma unidade industrial, bem como a utilização da foto em tempo real para rodar o sistema. 

Para melhorar a assertividade ainda mais dos objetos, seria interessante testar mais opções de treinamentos para os modelos da rede neural. Para isso, são necessários testes extensivos. Os tópicos abaixo provavelmente ajudariam a aumentar a acurácia do modelo:

\begin{itemize}
    \item Aumentar o \textit{dataset} de treinamento e validação dos modelos utilizando apenas imagens originais. (Figura \ref{fig:imagemBase})
    \item Criar as \textit{bounding boxes} utilizando o LabelImg [0,1,2,3,4,5,6,7,8,9] direto nas imagens originais + imagens com \textit{data augmentation};
\end{itemize}
