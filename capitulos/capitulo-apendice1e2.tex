\chapter{Sistema de identificação} \label{ap:ident}

Neste apêndice encontram-se algumas partes importantes dos códigos do método de \textit{Data Augmentation}, do Sistema de Identificação de Código de barras e de Identificação de Números. Os códigos inteiros podem ser encontrados através do \textit{link}:\url{https://github.com/AntonioHenriqueAC/Barcode_Number-Recognition/tree/master/google_colab}
\newline

\begin{lstlisting}[caption=Método de \textit{Data augmentation}]
import os
import random
from scipy import ndarray

# image processing library
import skimage as sk
from skimage import transform
from skimage import util
from skimage import io

def random_rotation(image_array: ndarray):
    # pick a random degree of rotation between 25% on the left and 25% on the right
    random_degree = random.uniform(-10, 10)
    return sk.transform.rotate(image_array, random_degree)

def random_noise(image_array: ndarray):
    # add random noise to the image
    return sk.util.random_noise(image_array)

def horizontal_flip(image_array: ndarray):
    # horizontal flip doesn't need skimage, it's easy as flipping the image array of pixels !
    return image_array[:, ::-1]

# dictionary of the transformations we defined earlier
available_transformations = {
    'rotate': random_rotation,
    'noise': random_noise,
    'horizontal_flip': horizontal_flip
}


from google.colab import drive
drive.mount('/content/drive')


folder_path = "PATH DAS IMAGENS ORIGINAIS"
num_files_desired = 1000

# find all files paths from the folder
images = [os.path.join(folder_path, f) for f in os.listdir(folder_path) if os.path.isfile(os.path.join(folder_path, f))]

num_generated_files = 0
while num_generated_files <= num_files_desired:
    # random image from the folder
    image_path = random.choice(images)
    # read image as an two dimensional array of pixels
    image_to_transform = sk.io.imread(image_path)
    # random num of transformation to apply
    num_transformations_to_apply = random.randint(1, len(available_transformations))

    num_transformations = 0
    transformed_image = None
    while num_transformations <= num_transformations_to_apply:
        # random transformation to apply for a single image
        key = random.choice(list(available_transformations))
        transformed_image = available_transformations[key](image_to_transform)
        num_transformations += 1

        new_file_path = '%s/augmented_image_%s.jpg' % (folder_path, num_generated_files)

        # write image to the disk
        io.imsave(new_file_path, transformed_image)
        num_generated_files += 1
\end{lstlisting}

%----------------------------------------------------------------------------
\newpage

\begin{lstlisting}[caption=Treinamento do modelo de \textit{barcodes}]
# Conectando o drive ao colab
from google.colab import drive
drive.mount('/content/drive')
        
# Baixando as bibliotecas necessarias
!pip3 install tensorflow-gpu==1.13.1
!pip3 install imageai --upgrade

# Treinar novo Modelo !
# 1. Baixando a CNN pre-treinada
!wget https://github.com/OlafenwaMoses/ImageAI/releases/download/essential-v4/pretrained-yolov3.h5

# Carregando os arquivos do drive no colab
!unzip "/content/drive/My Drive/Colab/datasets/barcode.zip"

# 3. Treinando o modelo
from imageai.Detection.Custom import DetectionModelTrainer

trainer = DetectionModelTrainer()
trainer.setModelTypeAsYOLOv3()
trainer.setDataDirectory(data_directory="barcode")
trainer.setTrainConfig(object_names_array=["barcode"], batch_size=4, num_experiments=11, train_from_pretrained_model="pretrained-yolov3.h5")
trainer.trainModel()


# 4. Avaliando os modelos gerados
metrics = trainer.evaluateModel(model_path="barcode/models", json_path="/content/drive/My Drive/Colab/json/detection_config_barcode.json", iou_threshold=0.5, object_threshold=0.3, nms_threshold=0.5)
\end{lstlisting}

%----------------------------------------------------------------------------
\newpage

\begin{lstlisting}[caption=Funções de eliminação de \textit{Bounding Boxes} sobrepostas e duplicadas, label=ap:IoU]
	# determine the (x, y)-coordinates of the intersection rectangle
	xA = max(boxA[0], boxB[0])
	yA = max(boxA[1], boxB[1])
	xB = min(boxA[2], boxB[2])
	yB = min(boxA[3], boxB[3])
	# compute the area of intersection rectangle
	interArea = max(0, xB - xA + 1) * max(0, yB - yA + 1)
	# compute the area of both the prediction
	# rectangles
	boxAArea = (boxA[2] - boxA[0] + 1) * (boxA[3] - boxA[1] + 1)
	boxBArea = (boxB[2] - boxB[0] + 1) * (boxB[3] - boxB[1] + 1)
	# compute the intersection over union by taking the intersection
	# area and dividing it by the sum of prediction
	# areas - the interesection area
	iou = interArea / float(boxAArea + boxBArea - interArea)
	# return the intersection over union value
	return iou

def boxArea(boxA, boxB):
	boxAArea = (boxA[2] - boxA[0] + 1) * (boxA[3] - boxA[1] + 1)
	boxBArea = (boxB[2] - boxB[0] + 1) * (boxB[3] - boxB[1] + 1)
	return boxAArea/boxBArea

def removeDuplicatesLoop(detections):
	i=0
	k=0
	for i, detection in enumerate(detections):
		for k, detection in enumerate(detections):
					inter = bb_intersection_over_union(detections[i]["box_points"] , detections[k]["box_points"])
					if(inter > float(0.01) and inter < float(1)):
						if(boxArea(detections[i]["box_points"] , detections[k]["box_points"]) > 1):
							detections.pop(k)
						else:
							detections.pop(i)
							removeDuplicatesLoop(detections)
							
\end{lstlisting}

%----------------------------------------------------------------------------
\newpage
\begin{lstlisting}[caption=Aumento da \textit{bouding box} e recorte dos \textit{barcodes} em imagens individuais, label=ap:AumentaBbox]

from imageai.Detection.Custom import CustomObjectDetection
from prettytable import PrettyTable

if( len(filesIMG) == len(dirs) ):
  print("Nao ha mais imagem para treinar. Por favor insira mais imagens na pasta /Imagens_Originais")
else:
  detector = CustomObjectDetection()
  detector.setModelTypeAsYOLOv3()
  detector.setModelPath(modelPath)
  detector.setJsonPath(JsonPath)
  detector.loadModel()

  detections = detector.
                    detectObjectsFromImage(input_image = InputImage, 
                    output_image_path="barcode-detected-OriginalBoxes.jpg",
                    minimum_percentage_probability=30)
                    
  for e in detections:
    e['box_points'][0] = e['box_points'][0] - 50 ## x1
    e['box_points'][1] = e['box_points'][1] - 50 ## y1
    e['box_points'][2] = e['box_points'][2] + 50 ## x2
    e['box_points'][3] = e['box_points'][3] + 50 ## y2

  removeDuplicatesLoop(detections)

  header = PrettyTable(["Nome", "Porcentagem ", "box_points"])
  for detection in detections:
    header.add_row([detection["name"], detection["percentage_probability"], detection["box_points"]])

  img = PrettyTable(["Barcodes identificados"])
  img.add_row([len(detections)])
\end{lstlisting}

%----------------------------------------------------------------------------
\newpage

\begin{lstlisting}[caption=Recorte das \textit{bounding box} e exportação para uma pasta tags-images]
matplotlib inline
import numpy as np
import skimage.io
import matplotlib.pyplot as plt
import skimage.segmentation
import cv2

if( len(filesIMG) == len(dirs) ):
  print("Nao ha mais imagem para treinar. 
  Por favor insira mais imagens na pasta '/Imagens_Originais'")
else:
  Image_with_boxes = "barcode-detected-OriginalBoxes.jpg"

  img = skimage.io.imread(InputImage)
  bbox = []

  for e in detections:
      bbox.append(e['box_points'])

  for i, (x1,y1,x2,y2) in enumerate(bbox):

      if i in {0,1,2,3,4,5,6,7,8,9}:
        i = '00' + str(i)
      
      if(int(i) > 9 and int(i) < 100):
        i = '0' + str(i)

      img2 = cv2.rectangle(img, (x1,y1), (x2,y2), (0,255,0), 0)
      out = img[y1:y2, x1:x2]

      cv2.imwrite('barcode-detected-objects/barcode-00'+str(i)+'.jpg', out)

  cv2.imwrite('barcode-detected-New.jpg', img)

  print("Boxes recortados com sucesso!")
  print("..")
  print("..")
  print("..")
  print("Imagem = 'barcode-detected-New.jpg' gerada.")
\end{lstlisting}

%----------------------------------------------------------------------------
\newpage
\begin{lstlisting}[caption=Função para fazer a diferença entre distância dos números para deletar o de menor Acurácia, label=ap:removeNumber]
def removeDuplicates():
    i=0
    for detection in detections:
      if( i != len(detections)-1 ):
          
          Xa = (detections[i]['box_points'][0] + detections[i]['box_points'][2])/2
          Ya = (detections[i]['box_points'][1] + detections[i]['box_points'][3])/2
          Xb = (detections[i+1]['box_points'][0] + detections[i+1]['box_points'][2])/2
          Yb = (detections[i+1]['box_points'][1] + detections[i+1]['box_points'][3])/2

          Dab = ( (Xb-Xa)**2 + (Yb - Ya)**2 )**1/2
          if(Dab < float(10)):
              if(detections[i]["percentage_probability"] < detections[i+1]["percentage_probability"]):
                  detections.pop(i)
              else:
                  detections.pop(i+1)
      i=i+1
\end{lstlisting}

%----------------------------------------------------------------------------

\begin{lstlisting}[caption=Identificando os números nas imagens com \textit{barcodes} individuais, label=ap:identNumber]
from imageai.Detection.Custom import CustomObjectDetection
import json
from prettytable import PrettyTable

modelPath = "Colab/models/number/detection_model-ex-034--loss-0012.508.h5"
JsonPath = "Colab/json/detection_config_number.json"

detector = CustomObjectDetection()
detector.setModelTypeAsYOLOv3()
detector.setModelPath(modelPath)
detector.setJsonPath(JsonPath)
detector.loadModel()

avgCorrida = 0

for x in range(file_count_tag):

  i=0
  soma=0
  numero=''
  execucoes = 0

  if x in {0,1,2,3,4,5,6,7,8,9}:
    x = '00' + str(x)
  
  if(int(x) > 9 and int(x) < 100):
    x = '0' + str(x)

  detections = detector.detectObjectsFromImage(input_image="Colab/Corridas/Corrida_"+str(corrida)+"/tags_images/barcode-00"+str(x)+".jpg", 
                                          output_image_path="Colab/Corridas/Corrida_"+str(corrida)+"/tags_detected/digit-detected-"+str(x)+".jpg", 
                                          display_percentage_probability=False, 
                                          minimum_percentage_probability=60)
  def boxPoints(e):
    return e['box_points']

  detections.sort(key=boxPoints)

  menor = 100
  while(len(detections) > 13):
    execucoes = execucoes + 1
    if(execucoes>10000):
      for i, detection in enumerate(detections):
        if(menor > detections[i]["percentage_probability"]):
          menor = detections[i]["percentage_probability"]   
      for k, detection in enumerate(detections):
        if(detections[k]["percentage_probability"] == menor):
          detections.pop(k)
      break
    removeDuplicates()

  img = PrettyTable(["                     Imagem                    "])
  img.add_row([x])
  header = PrettyTable(["Numero", "Porcentagem ", "box_points"])
  alert = ''
  for i in range(len(detections)):
    if(detections[i]["percentage_probability"] < 99):
      detections[i].update({"alert" : "true"})
      alert = "true"
    else:
      detections[i].update({"alert" : "false"})
      alert = "false"
    header.add_row([detections[i]["name"], detections[i]["percentage_probability"], detections[i]["box_points"]])
    soma = soma + detections[i]["percentage_probability"]
    numero = numero + ''.join(detections[i]["name"])

  if(len(detections) != 0):
    avg = soma/len(detections)
  table = PrettyTable(["Numeros", "Quantidade", "Media"])
  table.add_row([numero, len(detections), avg])
  
  avgCorrida = avgCorrida + avg
  num=numero
  config = {
      "num": num,
      "id": x,
      "acc": avg,
      "alert": alert
    }
  detections[:0] = [config]

  with open('Colab/Corridas/Corrida_'+str(corrida)+'/tags_json/tag-'+ str(x) +'.json', 'w', encoding='utf-8') as f:
    json.dump(detections, f, ensure_ascii=False, indent=4)
\end{lstlisting}


%----------------------------------------------------------------------------
\newpage

\begin{lstlisting}[caption=Criando o arquivo config-x.json]
avgCorrida = avgCorrida/float(file_count_tag)

config = {
    "numCorrida": num,
    "Status": "Inadimplente",
    "acc": avgCorrida,
    "qte": file_count_tag
}

with open('Colab/Corridas/Corrida_'+str(corrida)+'/config_'+str(corrida)+'.json', 'w', encoding='utf-8') as f:
    json.dump(config, f, ensure_ascii=False, indent=4)
\end{lstlisting}


%----------------------------------------------------------------------------
\newpage

\chapter{Aplicação Web} \label{ap:web}

Neste apêndice encontram-se algumas partes importantes dos códigos do software web desenvolvido . Os códigos inteiros podem ser encontrados através do \textit{link}: \url{https://github.com/AntonioHenriqueAC/Barcode_Number-Recognition}
\newline


%----------------------------------------------------------------------------

\begin{lstlisting}[caption=Exemplo do código dos \textit{controllers}]
var path = require('path');
var rootPath = path.join(__dirname, '../../');
var fs = require('fs');
var CorridaBs = require('../business/corrida.bs');
const image2base64 = require('image-to-base64');
const util = require('util')
var multer = require('multer');

const readDirPromise = util.promisify(fs.readdir);
const renamePromise = util.promisify(fs.rename)
const readFilePromise = util.promisify(fs.readFile)
const writeFilePromise = util.promisify(fs.writeFile)


module.exports.list = async (req, res) =>{
	var corrida = new CorridaBs();

	await callListPage();
	async function callListPage() {
		await corrida.listHome();
		const data = await corrida.list();
			res.render('list', {
			corrida: data
			})
	}

}	

module.exports.detailPage = async (req, res) => { 
	var corrida = new CorridaBs(req);

		await callDetailPage();
		async function callDetailPage() {
			let detail = await corrida.list();
			await corrida.groupCorridaDetail(req);
			let data = await corrida.listTags(req);
			
			const position = req.body.id - 1

			detail = detail[position]
			res.render('corrida-detail',{
				corrida: data,
				tags: data.length,
				detail: detail,
				position: position
			})
		}
}


module.exports.checkCorrida = async (req, res) => {
	
	let corrida = new CorridaBs();
	let targetCorrida = req.body.position;
	let dir = rootPath + 'server/corridas_config/config/';

	async function getDirectories(path) {
		const files = await readDirPromise(path);
		const file = parseInt(targetCorrida) 
		let nameFile = files[file];

		return new Promise((resolve) => {
			resolve(path + nameFile);
		})
	}
	const fileName = await getDirectories(dir)

	
	await callCheckCorrida();
	async function callCheckCorrida() {
			let resultFull = await corrida.list();

			result = resultFull[targetCorrida]
			var corridaStatus = result.status;
		
			result.Status = "Despachada"

			await writeFilePromise(fileName, JSON.stringify(result))

			if (corridaStatus == "Despachada") {
				res.render('corrida-desp');
			} else {
				res.render('corrida-check', {
				corrida: result.nuMCorrida
				})
			}
	}

}

module.exports.deleteCorrida = async (req, res) => {
var corrida = new CorridaBs();

var dir = rootPath + 'server/corridas_config/config/';
var dirDelete = rootPath + 'server/corridas_config/deleted/';


async function getDirectories(path ) {
	const files = await readDirPromise(path);
	const file = parseInt(req.body.id) - 1
	let nameFile = files[file];

	return new Promise((resolve)=>{
		resolve(path+nameFile);
	})
}

 	const result = await corrida.list()

	let targetCorrida = parseInt(req.body.id) -1;
	let corridaTarget = result[targetCorrida]

	const fileName = await getDirectories(dir)

	let moveFile = async (file, dir2) => {
		let f = path.basename(file);
		let dest = path.resolve(dir2, f);
		await renamePromise(file, dest);
	};

	await moveFile(fileName, dirDelete);

	res.render('corrida-delete', {
			corrida: corridaTarget.nuMCorrida
			})
};


module.exports.showImage = async (req, res) => {
	target = req.body.id - 1
	position = parseInt(req.body.position) + 1

	if (target >= 0 && target < 10) target = '00' + target
	if (target > 9 && target < 100) target = '0' + target

	var path = rootPath + "server/corridas/Corrida_" + position + "/tags_images/barcode-00" + target + ".jpg"
	var numCorrida = rootPath + "server/corridas/Corrida_" + position + "/tags_json/tag-" + target + ".json"
	
	const img = await image2base64(path);
	let result = await readFilePromise(numCorrida, 'utf8');
	result = JSON.parse(result);
	numCorrida = result[0].num

	res.render('show-image', {
		id: target,
		img: img, 
		numCorrida: numCorrida,
		position: position
	})
}

module.exports.editImage = async (req, res) => {

	var corrida = new CorridaBs(req);
	
	var numCorrida = rootPath + "server/corridas/Corrida_" + req.body.id + "/tags_json/tag-" + req.body.target + ".json"
	
	let result = await readFilePromise(numCorrida, 'utf8');
	result = JSON.parse(result);
	result[0].num = req.body.num
	
	
	fs.writeFile(numCorrida, JSON.stringify(result), function writeJSON(err) {
		if (err) return console.log(err);
	});
	
	await callDetailPage();
	async function callDetailPage() {
		let detail = await corrida.list();
		await corrida.groupCorridaDetail(req);
		let data = await corrida.listTags(req);
		
		const position = req.body.id - 1
		detail = detail[position]
		res.render('corrida-detail', {
			corrida: data,
			tags: data.length,
			detail: detail,
			position: position
		})
	}
}

module.exports.scriptPy = async (req, res) => {
const path = rootPath + "server/images"
const filesLenght = await readDirPromise(path);

var storage = multer.diskStorage({
	destination: function (req, file, callback) {
		callback(null, './server/images');
	},
	filename: function (req, file, callback) {
		callback(null, file.fieldname + (filesLenght.length + 1) + '.jpg');
	}
});

var upload = multer({storage: storage}).single('img');

upload(req, res, function (err) {
	if (err) {
			return res.end("Error uploading file.");
		}
	});

let runPy = new Promise(function (success, nosuccess) {
	const {spawn} = require('child_process');
	const pyprog =	 spawn('python', [rootPath + 'recognize.py'], {shell: true}); // add shell:true so node will spawn it with your system shell.

	let storeLines = []; // store the printed rows from the script
	let storeErrors = []; // store errors occurred
	pyprog.stdout.on('data', function (data) {
		storeLines.push(data);
	});

	pyprog.stderr.on('data', (data) => {
		storeErrors.push(data);
	});

	 pyprog.on('close', () => {
		if (storeErrors.length) {
			nosuccess(new Error(Buffer.concat(storeErrors).toString()));
		} else {
			success(storeLines);
		}
	});
});

 runPy.then( () =>{
	 res.redirect('/');
	 }
 );

}
\end{lstlisting}


%----------------------------------------------------------------------------
\newpage